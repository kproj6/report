\documentclass[11pt,a4paper,titlepage,oneside]{report}

\usepackage[english]{babel}
\usepackage[utf8]{inputenc} % input encoding is UTF-8

\usepackage{graphicx}
\usepackage{tabularx}
\usepackage{subcaption}
\usepackage{natbib}

\usepackage{color}
\usepackage[unicode,pdftex]{hyperref}

\begin{document}

% Title page %%%%%%%%%%%%%%%%%%%%%%%%%%%%%%%%%%%%%%%%%%%%%%%%%%%%%%%%
\begin{titlepage}

\begin{figure}
\centering
\begin{subfigure}{.5\textwidth}
\centering
\includegraphics[width=0.8\textwidth]{img/logo_NTNU.png}\\
\end{subfigure}%
\begin{subfigure}{.5\textwidth}
\centering
\includegraphics[width=0.8\textwidth]{img/logo_SINTEF.jpg}
\end{subfigure}
\end{figure}

\begin{center}
{\LARGE \textbf{TDT4290 - Customer Driven Project}}
\vfill
{\Huge \textbf{Ocean forecast}}

\vspace{12pt}
{\LARGE \textbf{SINTEF}}

\vspace{30pt}
{\LARGE \textbf{Final report}}
\vfill
{\LARGE \textbf{Autumn 2014}}
\end{center}
\vfill
\begin{tabular*}{\textwidth}{@{\extracolsep{\fill}} l l}
\textbf{Group 6} & \textbf{Advisor} \\
Arve Nygård & Gleb Sizov \\
Anders Smedegaard Pedersen & \\
Emil Jakobus Schroeder & \\
Hans Kristian Henriksen & \\
Marco Radavelli & \\
Ondřej Hujňák & \\
Ruben Håskjold Fagerli & \\
\end{tabular*}

\end{titlepage}

% Empty page %%%%%%%%%%%%%%%%%%%%%%%%%%%%%%%%%%%%%%%%%%%%%%%%%%%%%%%%
\newpage
\thispagestyle{empty}
\mbox{}
\newpage

% Abstract %%%%%%%%%%%%%%%%%%%%%%%%%%%%%%%%%%%%%%%%%%%%%%%%%%%%%%%%%%
\begin{abstract}
Abstract
\end{abstract}

% Signatures %%%%%%%%%%%%%%%%%%%%%%%%%%%%%%%%%%%%%%%%%%%%%%%%%%%%%%%%
\thispagestyle{empty}
\begin{center}
{\large \textbf{Trondheim, \today}}\\
\vspace{2.5cm}
\begin{tabularx}{\textwidth}{@{\extracolsep{1cm}} X X }
\dotfill & \dotfill \\
~Arve Nygård & ~Anders Smedegaard Pedersen \\[1cm]
\dotfill & \dotfill \\
~Emil Jakobus Schroeder & ~Hans Kristian Henriksen \\[1cm]
\dotfill & \dotfill \\
~Marco Radavelli & ~Ondřej Hujňák \\[1cm]
\dotfill & \\
~Ruben Håskjold Fagerli & \\[1cm]
\end{tabularx}
\end{center}

% Table of contents %%%%%%%%%%%%%%%%%%%%%%%%%%%%%%%%%%%%%%%%%%%%%%%%%
\tableofcontents
\addtocontents{toc}{\protect\thispagestyle{empty}}

% List of figures %%%%%%%%%%%%%%%%%%%%%%%%%%%%%%%%%%%%%%%%%%%%%%%%%%%
\listoffigures
\addtocontents{lof}{\protect\thispagestyle{empty}}

% List of tables %%%%%%%%%%%%%%%%%%%%%%%%%%%%%%%%%%%%%%%%%%%%%%%%%%%%
\listoftables
\addtocontents{lot}{\protect\thispagestyle{empty}}

\pagenumbering{arabic}
\setcounter{page}{0}

% Main body %%%%%%%%%%%%%%%%%%%%%%%%%%%%%%%%%%%%%%%%%%%%%%%%%%%%%%%%%
%%%%%%%%%%%%%%%%%%%%%%%%%%%%%%%%%%%%%%%%%%%%%%%%%%%%%%%%%%%%%%%%%%%%%

\chapter{Introduction}
\section{TDT 4290 - Customer driven project}
The task is set forth in the subject TDT 4290 - Customer driven project at the Norwegian university of science and technology. The goal of the course is 
\begin{quote}
(...)to give the students a practical experience of carrying out all the phases of a typical customer guided IS/IT-project. \cite{TDT4290:Intro}
\end{quote}
The subject divides the students into random groups, and assigns each group an assignment. The assignments are real problems that businesses needs solved. 

Although the assignment is to follow the entire process of an IT-project, the focus is on the earlier phases of a project. Thus, an important part of the assignment is the work leading up to the implementation phase. There is obviously also an important focus on the implementation itself. Maintenance is however left out of the scope of the projects.

\section{SINTEF}
About SINTEF

\section{Ocean currents}
About currents, why they are important, who is interested etc.

\section{Assignment and scope}

\chapter{Planning}

\chapter{Tools and technology}
\section{Documents}

  \subsection{\LaTeX}
  \LaTeX~is a typesetting system and document markup language that became standard for scientific documents. It is easily expandable by thousands of different packages and can handle all aspects of scientific papers.

  We have chosen to use \LaTeX~for our report for two main reasons - \LaTeX~sources are easily readible and, because they are simple text files, they can be easily versioned by various version control systems. Second reason was focus on content, not on form. In \LaTeX~sources there is only very little information about exact view of the page. \LaTeX~itself during compile time chooses the best position of elements so it complies with all typographic norms.

  We have created a template in our shared space together with bibliography file. Everyone could then write his sections in custom environment that suited him the best while the current state of the report was always available to all members.

  \subsection{Google Docs}
  Google Docs \footnote{\url{https://docs.google.com}} is a web based office suite including a text editor, a spreadsheet program and a presentation program. All files created in these programs can easily be shared with colaborators. By sharing files the colaborators get access to view and edit the files. Editing and commenting on other's work. We decided to use Google Docs for all documents that did not require the advanced typesetting of \LaTeX  so we had a common platform for such documents.

\section{Project management}
  \subsection{Trello}
Trello is a web-based collaborative project management tool originally made by Fog Creek Software (New York, USA) \footnote{\url{http://www.fogcreek.com/}}. 
It's based on the Kanban method which has first implemented by Toyotain 1953 to be used in car production \cite{wiki-kanban}. It has since been modified to be used in several different industries. 
David J. Anderson formulated a model based on Kanban for knowlegde based work, specifically software development, where the team work incrementally pulling work from a queue \cite{da2004}. 
This work queue in represented by a card for each task wich is located in a "to-do" area on a board. Task may then be moved to a bin called for example "in progress" to indicate that the task is beeing work on. Finally, when the task is complete, the cars can be moved to a "done" bin. In Trello this is implemented by elements you can drag and drop between bins. As an administrator you have the possibility to define both bins and boards which makes Trello a very versatile tool.  
Trello is a freemium webservice which means that it is free to use but additional support and features can be accessed if you pay a fee. As we only needed the standard functions and thus used the free version.
  \subsection{Slack}
Slack is a webbased team communication tool founded by Stewart Butterfield
\cite{wiki_slack}. It offers text chat in different channels and integration with a number of different popular services used by development teams \footnote{\url{https://slack.com/integrations}}. This was useful to us since we needed to share information that might be more relevant to specific team members and also to have a single means of communication. Using Slack's integration with Trello and Github we would get notified when there where changes on these platforms aswell. Slack's posibillity to share files, or link to files on Google Documents also came in handy.

\section{Version control}
  \subsection{Git}

\section{Programming languages}
  \subsection{Java}
  \subsection{JavaScript}
  As stated by Davis Flanagan i JavaScript: the definitive guide "Javascript is part of the triad of technologies that all Web developers must learn". He continues to note that JavaScript specifies the behaviour of the web page \cite{fd11}. Along with specification of HTML5 and ECMAscript 6 (the standard name of JavaScript) the posibilities of what you can achieve with JavaScript has greatly improved. Since our assignment was to create a web based solution it seamed natural to use JavaScript as part of the solution. This was further supported by the excistence of open source libraries designed to make interactive maps which was relevant for our assginment.

\section{IDEs}

\chapter{Pre study}
In this chapter, we present the findings of the pre study. The pre study document was made as a separate document that was meant to be delivered to the customer independently of this report. Therefore, there are overlapping sections with the full report. These sections will not be given in this chapter, but are instead presented in their respective chapters of the full report. 

\chapter{Requirements}

\chapter {Pre-sprint work}

\chapter{Sprint 1}

\chapter{Sprint 2}

\chapter{Sprint 3}

\chapter{Final solution}

\chapter{Evaluation}


% Sources cited in the document
% uncomment when there are some citations, uncomment bibtex in Makefile
\bibliographystyle{plain}
\chapter{Bibliography}
\begin{flushleft}
    \nocite{leafletjs.com}
    \nocite{openlayers.org}
	\bibliography{report}
\end{flushleft}

% Appendixes
\appendix

\end{document}
