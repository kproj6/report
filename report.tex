\documentclass[11pt,a4paper,titlepage,oneside]{report}

\usepackage[english]{babel}
\usepackage[utf8]{inputenc} % input encoding is UTF-8

\usepackage{graphicx}
\usepackage{tabularx}
\usepackage{subcaption}
%\usepackage{natbib}

\usepackage{color}
\usepackage[unicode,pdftex]{hyperref}

\begin{document}

% Title page %%%%%%%%%%%%%%%%%%%%%%%%%%%%%%%%%%%%%%%%%%%%%%%%%%%%%%%%
\begin{titlepage}

\begin{figure}
\centering
\begin{subfigure}{.5\textwidth}
\centering
\includegraphics[width=0.8\textwidth]{img/logo_NTNU.png}\\
\end{subfigure}%
\begin{subfigure}{.5\textwidth}
\centering
\includegraphics[width=0.8\textwidth]{img/logo_SINTEF.jpg}
\end{subfigure}
\end{figure}

\begin{center}
{\LARGE \textbf{TDT4290 - Customer Driven Project}}
\vfill
{\Huge \textbf{Ocean forecast}}

\vspace{12pt}
{\LARGE \textbf{SINTEF}}

\vspace{30pt}
{\LARGE \textbf{Final report}}
\vfill
{\LARGE \textbf{Autumn 2014}}
\end{center}
\vfill
\begin{tabular*}{\textwidth}{@{\extracolsep{\fill}} l l}
\textbf{Group 6} & \textbf{Advisor} \\
Arve Nygård & Gleb Sizov \\
Anders Smedegaard Pedersen & \\
Emil Jakobus Schroeder & \\
Hans Kristian Henriksen & \\
Marco Radavelli & \\
Ondřej Hujňák & \\
Ruben Håskjold Fagerli & \\
\end{tabular*}

\end{titlepage}

% Empty page %%%%%%%%%%%%%%%%%%%%%%%%%%%%%%%%%%%%%%%%%%%%%%%%%%%%%%%%
\newpage
\thispagestyle{empty}
\mbox{}
\newpage

% Abstract %%%%%%%%%%%%%%%%%%%%%%%%%%%%%%%%%%%%%%%%%%%%%%%%%%%%%%%%%%
\begin{abstract}
Abstract
\end{abstract}

% Signatures %%%%%%%%%%%%%%%%%%%%%%%%%%%%%%%%%%%%%%%%%%%%%%%%%%%%%%%%
\thispagestyle{empty}
\begin{center}
{\large \textbf{Trondheim, \today}}\\
\vspace{2.5cm}
\begin{tabularx}{\textwidth}{@{\extracolsep{1cm}} X X }
\dotfill & \dotfill \\
~Arve Nygård & ~Anders Smedegaard Pedersen \\[1cm]
\dotfill & \dotfill \\
~Emil Jakobus Schroeder & ~Hans Kristian Henriksen \\[1cm]
\dotfill & \dotfill \\
~Marco Radavelli & ~Ondřej Hujňák \\[1cm]
\dotfill & \\
~Ruben Håskjold Fagerli & \\[1cm]
\end{tabularx}
\end{center}

% Table of contents %%%%%%%%%%%%%%%%%%%%%%%%%%%%%%%%%%%%%%%%%%%%%%%%%
\tableofcontents
\addtocontents{toc}{\protect\thispagestyle{empty}}

% List of figures %%%%%%%%%%%%%%%%%%%%%%%%%%%%%%%%%%%%%%%%%%%%%%%%%%%
\listoffigures
\addtocontents{lof}{\protect\thispagestyle{empty}}

% List of tables %%%%%%%%%%%%%%%%%%%%%%%%%%%%%%%%%%%%%%%%%%%%%%%%%%%%
\listoftables
\addtocontents{lot}{\protect\thispagestyle{empty}}

\pagenumbering{arabic}
\setcounter{page}{0}

% Main body %%%%%%%%%%%%%%%%%%%%%%%%%%%%%%%%%%%%%%%%%%%%%%%%%%%%%%%%%
%%%%%%%%%%%%%%%%%%%%%%%%%%%%%%%%%%%%%%%%%%%%%%%%%%%%%%%%%%%%%%%%%%%%%

\chapter{Introduction}
\section{TDT 4290 - Customer driven project}
The task is set forth in the subject TDT 4290 - Customer driven project at the Norwegian university of science and technology. The goal of the course is 
\begin{quote}
(...)to give the students a practical experience of carrying out all the phases of a typical customer guided IS/IT-project. \cite{TDT4290:Intro}
\end{quote}
The subject divides the students into random groups, and assigns each group an assignment. The assignments are real problems that businesses needs solved. 

Although the assignment is to follow the entire process of an IT-project, the focus is on the earlier phases of a project. Thus, an important part of the assignment is the work leading up to the implementation phase. There is obviously also an important focus on the implementation itself. Maintenance is however left out of the scope of the projects.

\section{SINTEF}
About SINTEF

\section{Ocean currents}
About currents, why they are important, who is interested etc.

\section{Assignment and scope}

\chapter{Planning}
This section will give an overview of our plans for the project. It will include time estimation, group organization, risk management and quality assurance.
\section{Project name}
The client, SINTEF Fisheries and Aquaculture, has given us the project labeled ”Ocean Forecast”. The task is to improve the actual system of storing and retrieving data for the monitored and predicted parameters of ocean.
\section{Project sponsor (customer)}
The customer is SINTEF Fisheries and Aquaculture (SINTEF Fiskeri og havbruk AS). 
SINTEF was founded in 1950 and it is nowadays organized into 8 divisions. The division of Fisheries and Aquaculture was founded in 1999 and represents technological expertise and industry knowledge in the utilization of renewable marine resources. Under the vision "Technology for a better society" it works for a knowledge-based bio marine industry. Its goal is to meet market demands for technological research and development on renewable marine resources.
\section{Partners including responsible third party providers}
SINTEF Fisheries and Aquaculture, NTNU
\section{Background for the project: software system development}
The customer owns a working solution, running on its own servers, for the delivery and analysis of marine tracked and predicted data, in form of maps with overlays and charts.
The background for this project is to optimize the actual, slow and memory-consuming solution, with a new one, and/or creating new front-end application using those data, to cover more use-cases.
\section{Measurement of project effects}
For our project to be considered a success, some predefined criterias must be met. Our product must be working according to the requirements and pass all tests in order to be a success. All requirements prioritized high and medium must be fulfilled. Requirements that have low priority are optional, but should be implemented if there is sufficient time.
In our case, quite open goals were requested by the customer, and no precisely quantifiable measurements were defined.
If working on improving the actual solution the measurements should be:
- The time it takes to visualize data for the end user should be “reasonably low” even with slow internet connection (like from Chile) or from a mobile device.
- The same displayed features as the actual working system are still available
If working on new use cases, the measurements should be:
- The same displayed features as the actual working system are still available
- New features, to cover more use-cases are added (mobile app, or new charts).
In both cases, an open source based solution is better than a commercial one.
\section{General terms}
The choice implementation language, platform and tools to use are up to us. The requirement is that the solution should run on a server.
There are non-functional requirements, in terms of improving performance and storage consumption, that should be accomplished.
\section{Planned workload}
Group members will aim to work 24-25h on average per week. We have no fixed schedules; group members are free distribute the workload between days and weeks as they see fit.
\section{Schedule of results}
We are using the Scrum project management framework and it’s therefore natural to have deliverables at the end of each sprint. Each sprint will result in a working prototype. Requirements are prioritized with numbers 1-5. The final prototype shall meet all requirements of priority 4 or higher.
\section{Concrete project work plan}
This section describes the specific layout of the project. This project follows the SCRUM methodology, as discussed in the next chapter. The first two weeks were spent planning, preliminary studies and writing the requirement specification. As a common practice, the project was divided into sprints, each lasting two weeks. The first sprint started after first customer meeting. At the end of the last sprint all the project goals should be accomplished.
  \subsection{Phases}
  Project Set-up

Project Planning

Project Prestudy Sprint

Sprint 1
Sprint 2

Pre-delivery of report (17th October)

Sprint 3
Sprint 4

Demo Planning
Project Demonstration
Project Delivery

\subsection{Activities}
Pre-planning: The pre-planning phase represents the team members getting to know each other and understanding the task. Also the first meeting with both advisor and costumer.
      
Planning: The planning phase is the stage where the actual work on the project begins. We determine which technologies will be used and how it all will be realized.
          
Documentation: The documentation phase represents time spent documenting the work effort, including implementation, research etc., and administrative tasks like meetings.
          
Implementation: The implementation phase represents time spent on the implementation the system. This includes the programming of the back-end and the scripting of the user interface.
          
Testing: The testing phase represents time spent testing the system. This includes integration testing, unit testing, functional testing etc.
          
Presentation: The presentation phase is the final presentation of the system and the delivery of the report.

\subsection{Milestones}
The workload in this project is divided into multiple tasks with their own milestones. These milestones indicate when these task should be completed, in order to avoid delays which can affect the progress of the rest of the project.

\subsection{Person-hours per activity and phase + lectures + project management}
TO BE ADDED

\section{Project Organization}
\subsection{Organizational diagram of how the group is organized}

\subsection{Roles}
Scrum master: Ondrej Huinak
Customer communication: Marco Radavelli
QA and testing responsible: Emil 
Documentation responsible: Hans Kristian
Advisor communication: Hans 
Front-end leader: Anders
Back-end leader: Arve
System architect: Ruben

\subsection{Responsibilities of the different roles}
Scrum master: make sure that we meet our goals. Talk to the customer and with the advisor. Make sure that the groups meetings have a structure, that finish on time.
Customer communication: send email, messages to customer
QA and testing: 

\subsection{Weekly schedule}
Mondays 2-3 pm - Advisor meeting
Mondays 3-4pm - Team meeting
Thursdays 12-2pm - Team meeting

For weeks 37 and 38 the advisor meeting will be Thursday 4pm.

\chapter{Tools and technology}
\section{Documents}

  \subsection{\LaTeX}
  \LaTeX~is a typesetting system and document markup language that became standard for scientific documents. It is easily expandable by thousands of different packages and can handle all aspects of scientific papers.

  We have chosen to use \LaTeX~for our report for two main reasons - \LaTeX~sources are easily readible and, because they are simple text files, they can be easily versioned by various version control systems. Second reason was focus on content, not on form. In \LaTeX~sources there is only very little information about exact view of the page. \LaTeX~itself during compile time chooses the best position of elements so it complies with all typographic norms.

  We have created a template in our shared space together with bibliography file. Everyone could then write his sections in custom environment that suited him the best while the current state of the report was always available to all members.

  \subsection{Google Docs}
  Google Docs \footnote{\url{https://docs.google.com}} is a web based office suite including a text editor, a spreadsheet program and a presentation program. All files created in these programs can easily be shared with colaborators. By sharing files the colaborators get access to view and edit the files. Editing and commenting on other's work. We decided to use Google Docs for all documents that did not require the advanced typesetting of \LaTeX  so we had a common platform for such documents.

\section{Project management}
  \subsection{Trello}
Trello is a web-based collaborative project management tool originally made by Fog Creek Software (New York, USA) \footnote{\url{http://www.fogcreek.com/}}. 
It's based on the Kanban method which has first implemented by Toyotain 1953 to be used in car production \cite{wiki-kanban}. It has since been modified to be used in several different industries. 
David J. Anderson formulated a model based on Kanban for knowlegde based work, specifically software development, where the team work incrementally pulling work from a queue \cite{da2004}. 
This work queue in represented by a card for each task wich is located in a "to-do" area on a board. Task may then be moved to a bin called for example "in progress" to indicate that the task is beeing work on. Finally, when the task is complete, the cars can be moved to a "done" bin. In Trello this is implemented by elements you can drag and drop between bins. As an administrator you have the possibility to define both bins and boards which makes Trello a very versatile tool.  
Trello is a freemium webservice which means that it is free to use but additional support and features can be accessed if you pay a fee. As we only needed the standard functions and thus used the free version.
  \subsection{Slack}
Slack is a webbased team communication tool founded by Stewart Butterfield
\cite{wiki_slack}. It offers text chat in different channels and integration with a number of different popular services used by development teams \footnote{\url{https://slack.com/integrations}}. This was useful to us since we needed to share information that might be more relevant to specific team members and also to have a single means of communication. Using Slack's integration with Trello and Github we would get notified when there where changes on these platforms aswell. Slack's posibillity to share files, or link to files on Google Documents also came in handy.

\section{Version control}
  \subsection{Git}
  Git is a distributed version control system developed in 2005 by Linus Torvalds and Linux development community \cite{ProGit}. Git was made to be small, fast and easy to use especially for code management as it's main purpose was versioning of Linux kernel source code. Nowdays is Git one of the most used versioning controls systems in software development thanks to it's open linence and powerfull features.

  We have chosen Git because some members already know it and are able to work efficiently with it. Another advantage is easy branching and distributed architecture that allows you to work offline. 
  
  We have created an organization on GitHub\footnote{\url{https://github.com}} with multiple repositories for different separate parts of our work - reports, server sources, client sources. We have chosen GitHub because it is well known git hosting server that offers advanced features and stability. Moreover some members already had accounts on GitHub and were familiar with interface, which shortened time needed to setup a working environment.

\section{Programming languages}
  \subsection{Java}
  \subsection{JavaScript}
  As stated by Davis Flanagan i JavaScript: the definitive guide "Javascript is part of the triad of technologies that all Web developers must learn". He continues to note that JavaScript specifies the behaviour of the web page \cite{fd11}. Along with specification of HTML5 and ECMAscript 6 (the standard name of JavaScript) the posibilities of what you can achieve with JavaScript has greatly improved. Since our assignment was to create a web based solution it seamed natural to use JavaScript as part of the solution. This was further supported by the excistence of open source libraries designed to make interactive maps which was relevant for our assginment.

\section{IDEs}

\chapter{Pre study}
In this chapter, we present the findings of the pre study. The pre study document was made as a separate document that was meant to be delivered to the customer independently of this report. Therefore, there are overlapping sections with the full report. These sections will not be given in this chapter, but are instead presented in their respective chapters of the full report. 

\chapter{Requirements}

\chapter {Pre-sprint work}

\chapter{Sprint 1}

\chapter{Sprint 2}

\chapter{Sprint 3}

\chapter{Final solution}

\chapter{Evaluation}


% Sources cited in the document
% uncomment when there are some citations, uncomment bibtex in Makefile
\bibliographystyle{plain}
\chapter{Bibliography}
\begin{flushleft}
	\bibliography{report}
\end{flushleft}

% Appendixes
\appendix

\end{document}
